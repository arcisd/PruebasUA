\documentclass{article}

\usepackage[
	estudiante=0, % usar '1' para versión estudiante.
]{pruebaua}

% ---------------------------------------------------------------------------------------------

% RESULTADOS, CRITERIOS E INSTRUCCIONES:

\def\resultados{
	\resultado{Modela matemáticamente el universo tridimensional aplicando vectores en el contexto de fuerza, movimiento y energía.}
}

\def\criteriouno{Explica el significado de las funciones seno, coseno y tangente.}
\def\criteriodos{Formula el vector posición, su módulo y su versor y aplica conceptos de ángulos directores.}
\def\criteriotre{Calcula producto punto y producto cruz (trabajo y torque).}
\def\criteriocua{Expresa sus ideas en forma oral y escrita de manera lógica y coherente.}

\def\criterios{
	\criterio{1}{\criteriouno}
	\criterio{2}{\criteriodos}
	\criterio{3}{\criteriotre}
	\criterio{4}{\criteriocua}
}

\def\instrucciones{
	\instruccion{Esta evaluación se realizará a través de Microsoft Teams.}
	\instruccion{Dispone de 120 minutos para contestar, recuerde enviar su cuestionario antes de cerrarlo.}
	\instruccion{Utilice el Chat de Microsoft Teams en caso que sea necesario contactar al docente.}
	\instruccion{No se aceptan consultas de contenido durante el desarrollo de la evaluación.}
	\instruccion{Fundamente y desarrolle detalladamente sus respuestas.}
}

% ---------------------------------------------------------------------------------------------

% INSTRUMENTO:

\begin{document}

\begin{prueba}[
	criterios=\criterios,
	curso=QYF200 CÁLCULO,
	docente=Diego ARCIS,
	exigencia=60,
	fecha=21 de marzo de 2022,
	instrucciones=\instrucciones,
	logo=logo.pdf,
	puntajegeneral=3,
	puntajemultiple=2,
	nivel=2,
	resultados=\resultados,
	titulo=Prueba 1,
]{PRUEBA}

\seccionmultiple

\subcriterios{
	\subcriterio{1}{\criteriouno}
	\subcriterio{3}{\criteriotre}
}

\preguntamultiple[
	respuesta=a
]{Pregunta de prueba inicial...}{Pregunta de prueba final...}{
	\alternativa{Alternativa 1...}
	\alternativa{Alternativa 2...}
	\alternativa{Alternativa 3...}
}{Respuesta de prueba... \punto{1}}

\preguntamultiple[
	respuesta=a
]{Pregunta de prueba inicial...}{Pregunta de prueba final...}{
	\alternativa{Alternativa 1...}
	\alternativa{Alternativa 2...}
	\alternativa{Alternativa 3...}
}{Respuesta de prueba... \punto{1}}

\secciondesarrollo

\subcriterios{
	\subcriterio{2}{\criteriodos}
	\subcriterio{4}{\criteriocua}
}

\preguntadesarrollo[
	espacio=3,
	puntaje=2,
]{Pregunta de prueba...}{ Continuación de prueba...}{Respuesta de prueba... \punto{1}}

\preguntadesarrollo[
	espacio=3,
	puntaje=2,
]{Pregunta de prueba...}{ Continuación de prueba...}{Respuesta de prueba... \punto{1}}

\preguntadesarrollo[
	espacio=3,
	puntaje=2,
]{Pregunta de prueba...}{ Continuación de prueba...}{Respuesta de prueba... \punto{1}}

\end{prueba}

\end{document}